\documentclass[titlepage]{scrartlc}

\title{SymAC -- A Symbolic AC-Domain Linear Circuit Simulator}
\subtitle{Documentation}
\author{Patrick Schulz}
\date{11.09.2018}

\begin{document}
    \maketitle
    \section{Introduction}
    In electronics, linear circuits play a big role. Even if the circuit is inherently non-linear in its function, like a mixer or an oscillator, it still can often be characterized (at least in some
    points) in a linear manner. Other circuits as amplifiers are designed to be linear. In order to analyze these circuits, small-signal version of the circuit are created and the node voltages and 
    currents are derived to get one or more transfer functions (among other useful stuff). For networks containing more than only some resistors and controlled-sources, especially for non-standard
    topologies, this analysis tends to be very error-prone and tedious. There are a lot of techniques helping the designer on making good approximations like estimating the bandwidth of the circuit.
    However, this still relies on the designer not making any mathematical errors. On the other hand, electronics are not doable without circuit simulators, such as spice and spectre. Why don't 
    designers use a good tool for \emph{symbolic} circuit analysis? Good question!

    This documents present \emph{SymAC}, a symbolic AC-domain linear circuit simulator, which gives the designer \emph{symbolic} results, such as $v_\mathrm{out} = -g_m r_o$ instead of $12.5$.
    \section{Usage}
        \subsection{Supported Devices}
        SymAC supports almost\footnote{Currently missing: Transformers} all linear devices. If you are missing one, it is propably implementable with the available components. Technically, not all
        devices are needed to build all linear networks, since elements as controlled sources, gyrators or transformers can easily be built from other components. Still, for convenience it would
        be nice to have these components. Since it is easy to add them to the simulator, I will probably do this some time, but this has a very low priority right now. If you're doing ,,regular'',
        low-voltage electronics (like integrated electronics), you will almost certainly be satisfied with the available devices. Here are the supported devices:
        \begin{itemize}
            \item Voltage and current sources
            \item Impedances: resistors, capacitors and inductors
            \item Controlled sources: voltage and current in- and output
            \item Ideal operational amplifier (single ended)
        \end{itemize}
        \subsection{Results browser}
        \subsection{Post-simplification of results}
    \section{Mathematical Background -- Modified Nodal Analysis (MNA)}
    \section{Implementation}
        \subsection{Implementation of the main engine}
        \subsection{Structure of the processing of results}
        \subsection{Simplification}
\end{document}
